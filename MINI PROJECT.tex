\documentclass[10pt]{article}
\usepackage[usenames]{color} %used for font color
\usepackage{amssymb} %maths
\usepackage{amsmath} %maths
\usepackage[utf8]{inputenc} %useful to type directly diacritic characters
\begin{document}
\begin{align*}\documentclass{report}
\usepackage{graphicx}
\begin{document}
\title{center}{\Large MINI PROJECT REPORT\par
QUIZ GAME\par
\author{center}BAHAR JAIN\par  %name
0801CS211031}\par              %enrollment no.
\vspace{1cm}
\section{\Large Objective}\par
{\large The main objective of my project "QUIZ GAME' is to create a quiz game. The user can enter his name and age.\newline The user can answer the general knowledge question and the score will be calculated.\par The program is created using C language.\newline It has 8 functions. The code include 690 line.\par }
\section{Function Description}
\vspace{1cm}
\begin{enumerate}
\item{\Large Main }\par
{\large This function allows the user to choose an option to proceed.\newline It lets user to start the game. The user can instructions for playing quiz game.\newline. The user can also see the score or quit the game.}
\vspace{1cm}
\item{\Large game}\par
{\large This function includes the information about the game and allows user to continue or go to main menu}
\vspace{1cm}
\item{\Large round 1}\par
{\large This function include questions of first round. The questions are\newline asked and the user can enter the answers. Three questions are asked in this round.}
\vspace{1cm}
\item{\Large round 2}\par
{\large This function asks the questions of second round.\newline 4 points increase on each right answer and 1 point is deducted on each wrong answer\newline 10 questions are asked in this round.}
\vspace{1cm}
\item{\Large home}\par
{\large This function let the user to take input for going to main menu or to restart the game.}
\vspace{1cm}
\item{\Large view score}\par
{\large This function calculates the score of the user.}
\vspace{1cm}
\item{\Large help}\par
{\large This function shows the instruction to play the quiz.\newline It tells about how the points are awarded in the game.}
\vspace{1cm}
\item{\Large personal details}\par
{\large This function takes name and age of the user as input.}
\end{enumerate}
\section{Output screenshots}

\includegraphics{Screenshot 2022-11-23 at 2.11.41 PM.png}\newline
\includegraphics{Screenshot 2022-11-23 at 2.11.53 PM.png}\newline
\includegraphics{Screenshot 2022-11-23 at 2.12.03 PM.png}\newline
\includegraphics{Screenshot 2022-11-23 at 2.12.13 PM.png}\newline
\includegraphics{Screenshot 2022-11-23 at 2.12.28 PM.png}\newline
\includegraphics{Screenshot 2022-11-23 at 2.12.36 PM.png}\newline
\includegraphics{Screenshot 2022-11-23 at 2.12.38 PM.png}\newline
\includegraphics{Screenshot 2022-11-23 at 2.10.27 PM.png}\newline
\includegraphics{Screenshot 2022-11-23 at 2.12.54 PM.png}\newline

\section{Debugging screenshots}

\includegraphics{Screenshot 2022-11-23 at 1.13.44 PM.png}\newline
\includegraphics{Screenshot 2022-11-23 at 1.13.48 PM.png}\newline
\includegraphics{Screenshot 2022-11-23 at 1.13.48 PM.png}\newline
\includegraphics{Screenshot 2022-11-23 at 1.17.39 PM.png}\newline

\end{document}\end{align*}
\end{document}